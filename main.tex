%% An AI based procedure to literature review: the case of Vaccine Supply Chain
%% Fabian Castaño1, Julio Goez2, Nubia Velasco3
%% 1 Mercado Libre
%% 2 NHH, Norwegian School of Economics, Department of Business and Management Science
%% 3 School of Management, Universidad de los Andes. nvelasco@uniandes.edu.co
 
%% 1.	Introduction: 
%% This research addresses both methodological and academic objectives. The first leverages the extensive use of Artificial Intelligence (AI) in academic research to enhance efficiency; and the second responds to the growing interest in Vaccine Supply Chain (VSC) research after COVID-19 pandemic. To achieve these objectives, this study employs mixed-methods approach, combining qualitative and quantitative research techniques, allowing a comprehensive analysis of VSC and the development of a conceptual framework supported by AI tools. The primary is to accelerate the literature review process, identify key concepts, and evaluate theories related to the research topic. The methodology is designed to ensure systematic, rigorous, and reliable outcomes. Furthermore, to ensure accuracy and relevance, the AI-generated insights will be validated by experts, ensuring that the process remains scientifically robust and reproducible.
%% 2.	Methods and preliminary results
%% The proposed review follows five stages: (i) Search query and literature selection, (ii) AI-assitend information extraction, (iii) expert review and quantitative assessment, (iv) iterative evaluation with AI tools, (v) categorization and data synthesis. 
%% The search was limited to Science Direct, Scopus and PubMed databases, and to papers between 2000 and July 2024 focused on VSC, Operations Management and Operational Research. This yield 219 papers. Next, Ryyan AI and SciSpace were employed to categorize the papers and extract key information as addressed problem, solution approach, methodology, findings, and practical implications. The extracted data underwent validation and a scoring-based evaluation limiting the review to 96 papers. Finally, VosViewer was used to perform a cluster analysis identifying six dominant themes: vaccine transportation, VSC, optimization, simulation, allocation. 
%% 3.	 Findings 
%% The use of a pure AI approach highlighted the limitation of the LLM. It worked well to extract general ideas, but not to extract papers contribution. Even though, there were elements that match the human extracted insights. These coincidences were of great value to refine the author's reasoning process for filtering and classifying papers. The panel of experts’ opinion is crucial to define the classification of the literature, who identified a possible classification of the literature following the criteria of objectives, application, and solution methods. FALTA HALLAZGOS EN RESEARCH EN VSC
%% 4.	Conclusions: 
%% This work contribution is twofold. First, we aim to provide a review of the literature on VSC focusing on the contributions of OR into the field, with a special angle on the lessons learned from COVID-19. Second, we tested an approach to the review process using AI to validate and clarify the capabilities of the technology and to what extent its contribution could complement human work. This systematic approach combines AI-driven efficiency with expert validation, ensuring a robust and insightful review of vaccine supply chain research. By integrating quantitative and qualitative methods, the study provides a comprehensive analysis while maintaining scientific rigor and reproducibility. FALTA UNA CONCLUSIÓN SOBRE LO QUE SE ENCONTRÓ EN VSC

\documentclass{optica-article}

\journal{opticajournal} % for journals or Optica Open

\articletype{Research Article}

\usepackage{lineno}
\linenumbers % Turn off line numbering for Optica Open preprint submissions.

\begin{document}

\title{An AI based procedure to literature review: the case of Vaccine Supply Chain}

\author{Fabian Castaño,\authormark{1} Julio Goez,\authormark{2,*} and Nubia Velasco\authormark{3}}
\address{\authormark{1} Applied Intelligence Team, Mercado Libre\\ \authormark{2}NHH, Norwegian School of Economics, Department of Business and Management Science\\ \authormark{3}School of Management, Universidad de los Andes.}

\section{Introduction}

This research addresses both methodological and academic objectives. The first leverages the extensive use of Artificial Intelligence (AI) in academic research to enhance efficiency; and the second responds to the growing interest in Vaccine Supply Chain (VSC) research after COVID-19 pandemic. To achieve these objectives, this study employs mixed-methods approach, combining qualitative and quantitative research techniques, allowing a comprehensive analysis of VSC and the development of a conceptual framework supported by AI tools. The primary is to accelerate the literature review process, identify key concepts, and evaluate theories related to the research topic. The methodology is designed to ensure systematic, rigorous, and reliable outcomes. Furthermore, to ensure accuracy and relevance, the AI-generated insights will be validated by experts, ensuring that the process remains scientifically robust and reproducible.

\section{Methods and preliminary results}

The proposed review follows five stages: (i) Search query and literature selection, (ii) AI-assitend information extraction, (iii) expert review and quantitative assessment, (iv) iterative evaluation with AI tools, (v) categorization and data synthesis.

The search was limited to Science Direct, Scopus and PubMed databases, and to papers between 2000 and July 2024 focused on VSC, Operations Management and Operational Research. This yield 219 papers. Next, Ryyan AI and SciSpace were employed to categorize the papers and extract key information as addressed problem, solution approach, methodology, findings, and practical implications. The extracted data underwent validation and a scoring-based evaluation limiting the review to 96 papers. Finally, VosViewer was used to perform a cluster analysis identifying six dominant themes: vaccine transportation, VSC, optimization, simulation, allocation.

\section{Findings}

The use of a pure AI approach highlighted the limitation of the LLM. It worked well to extract general ideas, but not to extract papers contribution. Even though, there were elements that match the human extracted insights. These coincidences were of great value to refine the author's reasoning process for filtering and classifying papers. The panel of experts’ opinion is crucial to define the classification of the literature, who identified a possible classification of the literature following the criteria of objectives, application, and solution methods \textbf{FALTA HALLAZGOS EN RESEARCH EN VSC}.

\section{Conclusions}

This work contribution is twofold. First, we aim to provide a review of the literature on VSC focusing on the contributions of OR into the field, with a special angle on the lessons learned from COVID-19. Second, we tested an approach to the review process using AI to validate and clarify the capabilities of the technology and to what extent its contribution could complement human work. This systematic approach combines AI-driven efficiency with expert validation, ensuring a robust and insightful review of vaccine supply chain research. By integrating quantitative and qualitative methods, the study provides a comprehensive analysis while maintaining scientific rigor and reproducibility. \textbf{FALTA UNA CONCLUSIÓN SOBRE LO QUE SE ENCONTRÓ EN VSC}

%%%%%%%%%% If using BibTeX:
\bibliography{sample}

%%%%%%%%%% If preparing manually:
% \begin{thebibliography}{1}
% \newcommand{\enquote}[1]{``#1''}

% \bibitem{Zhang:14}
% Y.~Zhang, S.~Qiao, L.~Sun, Q.~W. Shi, W.~Huang, L.~Li, and Z.~Yang,
%   \enquote{Photoinduced active terahertz metamaterials with nanostructured
%   vanadium dioxide film deposited by sol-gel method,}
%   {\protect\JournalTitle{Optics Express}} \textbf{22}, 11070--11078 (2014).

% \bibitem{Optica}
% {Optica}, \enquote{{Optica Publishing Group},}
%   \url{http://www.opg.optica.org}.

% \bibitem{FORSTER2007}
% P.~Forster, V.~Ramaswamy, P.~Artaxo, T.~Bernsten, R.~Betts, D.~Fahey,
%   J.~Haywood, J.~Lean, D.~Lowe, G.~Myhre, J.~Nganga, R.~Prinn, G.~Raga,
%   M.~Schulz, and R.~V. Dorland, \enquote{Changes in atmospheric consituents and
%   in radiative forcing,} in \enquote{Climate Change 2007: The Physical Science
%   Basis. Contribution of Working Group 1 to the Fourth Assesment Report of
%   Intergovernmental Panel on Climate Change,}  S.~Solomon, D.~Qin, M.~Manning,
%   Z.~Chen, M.~Marquis, K.~B. Averyt, M.~Tignor, and H.~L. Miler, eds.
%   (Cambridge University Press, 2007).

% \end{thebibliography}

\end{document}




